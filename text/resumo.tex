

O projeto ProInfoData monitora diariamente os computadores de todas as escolas públicas do Brasil. 

O monitoramento visa disponibilizar dados para que o MEC e a sociedade acompanhem o estado
de funcionamento dos computadores. 

Com o crescimento do parque computacinal das escolas e consequente aumento no volume de dados gerados,
a arquitetura original de armazenamento e consulta de dados, que é baseada em um modelo 
relacional com armazém de dados (Data Warehouse), já não se mostra eficiente.

Para melhorar a performance do sistema visamos uma solução que utiliza MapReduce, que é uma tecnologia emergente,
mas que mostrou-se eficiente em diversas implementações.

A solução que propomos com essa monografia é a transformação do modelo relacional,
atualmente empregado no projeto ProIndoData, para um modelo "chave-valor".