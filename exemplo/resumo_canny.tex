\documentclass[a4paper,12pt]{article}
\usepackage[top=2cm,left=4cm,right=4cm,bottom=2cm]{geometry}
\RequirePackage[brazil]{babel}
\usepackage[brazil]{babel}
\usepackage[utf8]{inputenc}
\usepackage{cite}
\usepackage{verbatim}
\usepackage{latexsym}
\usepackage{amsfonts}
\usepackage{amssymb}
\usepackage{graphicx}
% Title Page
\title{Resumo do artigo de John Canny: \\ 
  A Computational Approach to Edge Detection
        \newline 
        \newline 
        Bacharelado em Ciência da Computação \\ 
        CI394 - Processamento de Imagens}

\author{Professora Olga Regina Pereira Bellon\\ Tiago Rodrigo Kepe}

\begin{document}
\maketitle

\section{\textbf{Introdução}}

Canny\cite{CANNY} definiu um conjunto de requisitos que pode ser
compartilhado por qualquer aplicação de detecção de bordas, conseguiu
simplificar a análise de imagens e reduziu a quantidade de dados a serem
processados. 

Para isso, primeiramente foram eliminados os ruídos da imagem
com um filtro Gaussiano, após é calculado o gradiente e a direção das
bordas, as quais são discretizadas para uma direção que pode ser traçada
na imagem, por último são aplicados os métodos de supressão não máxima e o
de histerese.

\section{\textbf{Desenvolvimento}}

Canny\cite{CANNY} obteve suceso em sua abordagem por ter definido bons parâmetro para
detecção de bordas. Dois desses são definidos matematicamente para
detecção de borda, o critério de detecção e o de localização. Um terceiro
critério garante que o detector tenha apenas uma resposta para cada
borda. O algoritmo é executado em passos para a detecção final da borda.

John Canny\cite{CANNY} extendeu o detector ótimo, que utiliza uma aproximação onde
as bordas são marcadas com o máximo do gradiente de uma imagem
suavizada com um filtro Gaussiano, com operadores que suprimem os ruídos.

Uma borda em uma imagem pode apontar em diversos sentidos, de modo que o
algoritmo de Canny usa quatro filtros para detectar bordas horizontal, 
vertical e diagonal na imagem. O ângulo de direção da borda é arredondado
para um dos quatro ângulos que representam vertical, horizontal e as duas
diagonais (0, 45, 90 e 135 graus, por exemplo).

Dadas as estimativas dos gradientes da imagem, uma pequisa é conduzida para
determinar se a magnitude do gradiente assume um máximo local na direção
do gradiente. Uma otimização numérica é usada para encontrar operadores ótimos
que encontram cumes de bordas. Ele definiu operadores para os passos de
detecção de borda, e como resultado foi gerada uma classe de operadores.

Propôs um processo de afinamento de bordas conhecido como supressão não
máxima que utiliza uma máscara 3X3, onde o ponto central é considerado
se for maior que os demais pontos de interesse dessa máscara.

Também desenvolveu um outro processo conhecido como histerese, cuja função é a de
eliminar a fragmentação das bordas causada pelo ruído da imagem.

Quando esses processos estiverem concluídos, temos uma imagem binária onde cada
pixel é marcado como pixel de borda ou não.

\section{\textbf{Variantes do método de Canny}}
    
    \begin{itemize}
        \item Miranda e Camargo\cite{VAR1} aperfeiçoaram o método de Canny para detecção de
        bordas em imagens de intra-regiões, eles utilizaram o método de difusão
        anisotrópica, o qual tem como idéia básica a filtragem do espaço-escala,
        proporcionando representação da imagem em múltiplas escalas, comparado
        com o método Canny eles obtiveram o seguinte resultado:
            
        \includegraphics[width=80px,height=80px]{img/img_canny1.png}
    
        \textbf{Figura 1 Bordas de Canny da imagem original.}

        \includegraphics[width=80px,height=80px]{img/img_canny2.png}
        
        \textbf{Figura 2 Bordas de Canny da imagem filtrada por difusão
            anisotrópica com 20 iterações.}


        \item Torreão\cite{VAR2} localizou filtros que obtiveram desempenho
        superior ao proposto por Canny. Este filtro é baseado em funções
        de Green de equações de casamento de imagens, as quais foram
        construídas através dos operadores diferenciais D1, D2 e D3.

        \item Rodrigues\cite{VAR3} procurou um filtro que obtivesse
        desempenho superior ao proposto por Canny. Baseou-se no trabalho de
        Torreão\cite{VAR2} que propôs um filtro baseado em funções de Green.
        Dentro desse contexto, construiu outras combinações de operadores
        diferenciais, utilizando operadores de Torreão, criando e implementando,
        assim, novos filtros de detecção de bordas, com isso, conseguiu bordas
        mais suaves, como poderemos notar na figura:

        \includegraphics[width=400px,height=200px]{img/img_canny3.png}

        \textbf{Comparação visual entre o filtro de Canny (esquerda) e o
            filtro D13 (direita)}


        \item Richa\cite{VAR4} propôs um método para rastreamento do
        ventrículo esquerdo em imagens de ressonância magnética, ele
        utilizou a detecção de bordas de Canny, porém sem utilizar as etapas
        de supressão não máxima e histerese, pois, segundo ele, o gradiente
        guarda várias características da imagem original que são perdidas
        nesses processos, dentro do contexto do artigo.

    \end{itemize}

\section{\textbf{Conclusão}}

O algoritmo de Canny\cite{CANNY} é adaptável a diversos ambientes. Seus
parâmetros permitem que ele seja adaptado para o reconhecimento das bordas de 
diferentes características, dependendo dos requisitos específicos de uma
determinada implementação. Ele definiu um algoritmo para detecção de borda
em qualquer perfil de problema. Baseou-se em critérios de boa detecção e
localização em uma forma matemática. Propôs um método eficiente para
a geração de máscaras altamente direcionais em várias orientações. 

\bibliographystyle{plain}
\bibliography{resumo_canny}
\end{document}
