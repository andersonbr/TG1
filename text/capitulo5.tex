\chapter{\textbf{Discussão}}

Pode-se notar que a implementação das funções Map e Reduce é bastante
simples, permitindo que o programador se atenha apenas a geração dos
dados, sem preocupar-se com a comunicação entre os nós.

A simplicidade na codificação MapReduce e no armazenamento do
Hadoop, torna a transformação mais fácil, pois é necessário apenas associar elementos dos dois modelos.
No nosso caso ocorrerão as seguintes transformações: 
tabelas serão mapeadas em arquivos, tuplas corresponderão a linhas dos arquvos, os atributos a campos dos arquivos, 
as funções de carregamento corresponderão as funções Map, as funções de agregação
(carregamento dos DMs) funções Reduce, e as consultas serão realizadas pelo Hive.

Pontos importantes ainda a sererem estudados são as consultas feitas
pelo Hive, e a migração dos dados da base do ProInfoData. Mas essa parte
será focada na continuação desse trabalho.
